
\begin{spacing}{1.2}
  \chapter{PENDAHULUAN}
\end{spacing}

\pagenumbering{arabic}
\vspace{4ex}

Penelitian ini memiliki beberapa latar belakang yang disebabkan oleh berbagai kondisi yang menjadi 
acuan. Selain itu juga terdapat beberapa permasalahan yang akan dijawab sebagai keluaran dari 
penelitian.

\section{Latar Belakang}
Virus Corona merupakan varian virus yang dapat menginfeksi manusia, gejala yang ditimbulkan oleh varian
virus ini meliputi infeksi paru-paru, pneumonia, demam, dan kesusahan bernapas. Dalam 2 dekade terakhir,
Virus Corona telah menyebabkan 3 pandemi besar \cite{10.1093/ajcp/aqaa029}.
Pada tahun 2002, sebuah varian virus corona yang dinamakan \textit{Severe Acute Respiratory Syndrome Corona Virus
(SARS-CoV)} menginfeksi manusia untuk pertama kalinya di Cina \cite{pmid12690091,pmid12690092},
varian virus ini menginfeksi 8.000 orang dan menyebabkan kematian bagi 750 orang. 10 Tahun kemudian pada tahun 2012,
varian baru virus corona yang dinamakan \textit{Middle East Repiratory Syndrome Corona Virus (MERS-CoV)} \cite{pmid23075143}.
MERS-COV menginfeksi 2.494 pasien dari beberapa negara di Timur Tengah dan menimbulkan korban jiwa sebanyak 866 orang. Kemudian varian Virus Corona
yang dinilai paling mematikan muncul pada pada Desember 2019 dan menyebabkan pandemi besar di Kota Wuhan, Provinsi Hubei,
Cina. Virus ini dinamakan SARS-CoV-2.

Penyakit yang ditimbulkan oleh SARS-CoV 2 dinamakan \textit{Corona Virus disease 2019} atau COVID-19,
gejala COVID-19 meliputi sakit kepala berat, demam, batuk, dan diare \cite{pmid32835023}.
Pada 29 Oktober 2021, COVID-19 telah menginfeksi lebih dari 450 juta orang dan mengakibatkan kurang lebih 6.000.000 korban jiwa
\cite{WHO}. Walaupun tingkat persentase kematian SARS-CoV 2 lebih rendah dibandingkan SARS-CoV maupun MERS-CoV,
kemampuan penyebaran SARS-CoV 2 lebih tinggi dibandingkan varian virus sebelum-sebelumnya \cite{pmid32226295}.
Virus SARS-CoV 2 dapat dengan mudah menyebar melalui \textit{droplet} atau tetesan air yang dikeluarkan dari
penderita. Penularan virus SARS-COV 2 sering terjadi kepada orang dengan kondisi sehat
saat penderita berbicara tanpa menggunakan masker kepada orang tersebut \cite{pmid32355904}.

Tidak hanya kerugian dalam dampak kesehatan, pandemi Covid-19 juga memberikan dampak ekonomi secara negatif dimana
bisnis pada berbagai sektor mengalami penurunan keuntungan \cite{pmid34337569}.
Sepanjang pandemi Covid-19, negara di seluruh dunia pun mengalami penurunan
pemasukan.  Rata-rata penurunan pemasukan tersebut bernilai 4.8\% pada negara dengan pemasukan kecil dan penurunan rata-rata
8.9\% pada negara dengan pemasukan besar. Salah satu contoh sektor yang mengalami kerugian besar ialah sektor aviasi. Penerbangan sektor internasional
menjadi terhambat dikarenakan negara-negara yang melakukan \textit{lockdown} dan menerapkan kebijakan
\textit{travel restriction}. Hal ini dibuktikan dengan penerbangan komersial Internasional Amerika
yang awalnya 4.000 penerbangan per bulan pada tahun 2019 turun menjadi 1.500 penerbangan per
bulannya pada tahun 2020 \cite{TRUONG2021102126}.

Mudahnya penularan virus SARS-CoV 2 serta besarnya efek negatif yang ditimbulkan menjadi alasan bagi pemerintah untuk
melakukan pencegahan dan mengurangi penularan virus SARS-CoV 2. \textit{World Health Organization}
menyarankan agar masyarakat menjaga jarak dan melakukan \textit{physical distancing}
dengan jarak satu meter \cite{WHO2}. Berdasarkan hal tersebut, Indonesia melakukan berbagai
kebijakan, beberapa diantaranya ialah pengadaan PPKM yang diatur pada PerPres no 21 tahun 2020 yang meliputi
peliburan sekolah dan tempat kerja, pembatasan kegiatan keagamaan, dan pembatasan kegiatan di tempat umum.
Selain kebijakan tersebut Kepala Kepolisian Negara Republik Indonesia juga mengeluarkan maklumat
yang melarang adanya aktivitas berkumpulnya massa dalam jumlah banyak, baik di tempat umum maupun di lingkungan
sendiri \cite{DJALANTE2020100091}.

Dalam melakukan pengawasan kerumunan dan mengurangi penularan virus corona, sering kali petugas
melakukan patroli dan peninjauan lokasi secara langsung. Namun patroli tentunya membutuhkan personil untuk datang ke 
lokasi dan hanya dilakukan pada jam serta hari-hari tertentu. Tidak hanya itu, patroli yang dilakukan tidak dapat
mendata tempat keramaian secara detail untuk analisa lebih lanjut. Dengan permasalahan tersebut,
penulis menawarkan solusi untuk melakukan pengawasan kerumunan dengan pembelajaran mesin menggunakan
algoritma mask R-CNN. Selain untuk pengawasan kerumunan untuk mengurangi penyebaran Covid-19, data
deteksi kerumunan ini juga dapat digunakan untuk desain arsitektur tempat umum seperti ruang tunggu,
maupun pintu masuk ruangan yang lebih mendukung gaya hidup pasca Covid-19.

\section{Rumusan Masalah}
Pemantauan kerumunan selama ini masih dijalankan secara pemeriksaan manual. Petugas masih harus melakukan
pemantauan pada video cctv atau datang ke lokasi langsung untuk melakukan patroli. Jika kondisi wilayah yang
perlu diamati bertambah dan sumber daya manusia pemerintah kota tidak mencukupi dalam melaksanakan pemantauan kerumunan di
berbagai wilayah, maka kerumunan tidak dapat diketahui dan berpotensi besar menjadi salah satu media penyebaran
Virus Corona. Berdasarkan penjelasan di atas, maka akan dikembangkan sebuah sistem deteksi kerumunan otomatis
menggunakan algoritma kecerdasan buatan Mask R-CNN. Harapannya tidak hanya data deteksi kerumunan yang diberikan
dapat digunakan untuk mengatasi dan melaporkan situasi kerumunan,
data ini juga dapat digunakan dalam pemodelan penularan virus maupun desain ruang yang lebih baik.

\section{Tujuan}
Tujuan utama dari penelitian ini adalah mempersiapkan teknologi untuk melakukan
pengamatan secara otomatis terhadap suatu wilayah dalam rangka mengurangi penyebaran virus
corona maupun mengambil informasi mengenai kerumunan di beberapa wilayah,
serta mengurangi sumber daya manusia yang diperlukan untuk melakukan pengawasan kerumunan.
Selain untuk mempersiapkan teknologi masa depan yang dapat memonitoring kerumunan, 
deteksi kerumunan secara otomatis juga dapat digunakan untuk melihat apakah PPKM yang 
dilaksanakan berhasil atau tidak dengan mengetahui jumlah kerumunan yang terjadi. 
Semua hal ini dilakukan dalam rangka menurunkan kasus Covid-19 dan mengurangi penyebaran Virus Corona.

\section{Batasan Masalah}

Untuk memfokuskan permasalahan yang diangkat, maka dilakukan pembatasan masalah.
Batasan - batasan masalah tersebut diantaranya adalah :

\begin{enumerate}[itemsep=-0.2em]
      \item Data yang diambil merupakan data video dengan kondisi nyata di salah satu pusat
            perbelanjaan Surabaya, Jawa Timur, Indonesia.
      \item Algoritma yang digunakan dalam penelitian ini ialah Mask R-CNN.
      \item Pengambilan data dilakukan dari ketinggian dan sudut pengambilan yang akan didefinisikan pada Bab 3.
      \item Kelas yang dideteksi dibedakan menjadi tiga kelas yaitu \textit{person, group, crowd} yang didefinisikan oleh peneliti.
      \item Pengambilan data dilakukan pada bulan Januari 20222 di Pasar Atom Surabaya pada jam 1 siang hingga 4 sore.
      \item Objek yang dideteksi merupakan objek orang segala usia.
      \item Hasil keluaran dari penelitian ini adalah model pembelajaran mesin yang dapat digunakan untuk mendeteksi kerumunan Indonesia.

\end{enumerate}

\section{Manfaat}
Diharapkan penelitian ini dapat mendeteksi kerumunan dalam rangka membantu penanganan Covid-19 
maupun pandemi masa depan yang mungkin terjadi. Hasil dari deteksi kerumunan dapat digunakan sebagai
sistem \textit{monitoring} atau sebagai penghasil data mengenai perilaku masyarakat pasca pandemi Covid-19,
selain itu, diharapkan penelitian ini dapat membantu penanganan pandemi di masa depan.

\section{Sistematika Penulisan}

Laporan penelitian ini tersusun dalam sistematika dan terstruktur sehingga mudah
dipahami dan dipelajari oleh pembaca maupun seseorang yang ingin melanjutkan penelitian ini.
Alur sistematika penulisan laporan penelitian ini yaitu :

\begin{enumerate}[topsep=0em]

      \item \textbf{BAB I Pendahuluan}
            Bab ini berisi uraian tentang latar belakang permasalahan, penegasan
            dan alasan pemilihan judul, sistematika laporan, tujuan dan metodologi penelitian.
      \item \textbf{BAB II Tinjauan Pustaka}

            Bab ini berisi tentang uraian secara sistematis teori - teori yang
            berhubungan dengan permasalahan yang dibahas pada penelitian ini.
            Teori - teori ini digunakan sebagai dasar dalam penelitian, yaitu informasi terkait
            \textit{Deep Learning}, \textit{Object Detection},
            \textit{Mask R-CNN}, dan teori - teori penunjang lainnya.

      \item \textbf{BAB III Desain dan Implementasi Sistem}

            Bab ini berisi tentang penjelasan - penjelasan terkait eksperimen yang akan dilakukan,
            langkah - langkah pengambilan dataset dan proses pembagian kelas kerumunan.
            Untuk mendukung hal tersebut, maka ditampilkan pula \textit{workflow} agar proses penelitian
            yang akan dibuat dapat terlihat dan mudah dibaca untuk proses pembuatan pada pelaksanaan tugas akhir.

      \item \textbf{BAB IV Pengujian dan Analisa}

            Bab ini menjelaskan tentang hasil serta analisis yang didapatkan dari pengujian yang
            dilakukan mulai dari hasil pengujian \textit{precision}, \textit{recall}, \textit{Confusion Matrix}
            serta rekomendasi penggunaan model.

      \item \textbf{BAB V Penutup}

            Bab ini berisi penutup yang berisi kesimpulan yang diambil dari penelitian dan
            pengujian yang telah dilakukan. Saran mengenai langkah selanjutnya untuk pengembangan
            lebih lanjut juga dituliskan pada bagian ini.

\end{enumerate}