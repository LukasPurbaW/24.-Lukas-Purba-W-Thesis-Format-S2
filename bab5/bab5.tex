\chapter{PENUTUP}
\label{sec:chap5_tutup}
\vspace{1ex}
\section*{}

\section{Kesimpulan}
\label{sec:sec4_kesimpulan}
\vspace{1ex}
Pada penelitian ini, sebuah dataset mengenai aktivitas jual beli pada suatu pusat perbelanjaan Di
Surabaya berhasil dibentuk dengan total gambar 730 dan total objek 4.980.
Dataset yang dibentuk memiliki total objek kelas \textit{person} sebanyak 3.382 objek,
objek kelas \textit{group} sebanyak 1.350 objek, dan objek kelas \textit{crowd} sebanyak 248 objek.
Dataset tersebut dapat diakses pada \href{https://www.kaggle.com/datasets/lukaspurbaw/indonesia-pasar-atom-crowd-dataset}. 
Selain dataset, penelitian ini juga memberikan kontribusi berupa alur kerja penerapan deteksi objek
kerumunan pada pusat perbelanjaan, dan memberikan model yang dapat mendeteksi kerumunan dengan baik.
Berdasarkan hasil penilitian ini penggunaan algoritma Mask R-CNN pada dataset ini terbukti dapat 
membedakan objek kelas \textit{person} dimana objek kelas \textit{person} adalah orang yang tidak berdekatan dengan orang lain
dalam jarak kurang lebih satu meter dengan objek kelas \textit{group} dan kelas \textit{crowd} yang
merupakan objek orang yang saling berdekatan dalam jarak satu meter.  Model Mask R-CNN yang dibentuk
dengan \textit{backbone} ResNet 101 memiliki akurasi 88,79\% dalam mendeteksi objek pada dataset dengan
parameter \textit{confidence} lebih dari sama dengan 70\% dan \textit{intersect over union}
lebih dari sama dengan 50\% dengan \textit{mask} asli meskipun terdapat ketidak seimbangan data. Model yang dibentuk masih memiliki nilai \textit{recall}
dan presisi yang kecil dalam mendeteksi objek kelas \textit{crowd} karena data kelas \textit{crowd}
hanyalah 5\% dari total seluruh objek yang ada. Secara umum, model yang dibentuk dapat memprediksi 
adanya \textit{group} dan \textit{crowd}. Namun apakah model ini dapat digunakan dalam tempat
pusat perbelanjaan lain masih perlu diteliti lebih lanjut.

\section{Saran}
\label{sec:sec4_saran}
\vspace{1ex}
Walaupun model ini secara umum sudah dapat mendeteksi apakah terdapat kerumunan atau tidak, namun
nilai \textit{recall} dan presisi saat mendeteksi objek \textit{crowd} masih dinilai rendah. Hal ini
disebabkan oleh tidak seimbangnya dataset yang telah dikumpulkan. Dikarenakan pengumpulan dataset
dilakukan pada masa pandemi Covid-19, maka jumlah kerumunan yang terjadi pun tidak banyak. Dataset
dan model dapat diperbaiki dengan menambah data terutama objek kelas \textit{crowd}.

