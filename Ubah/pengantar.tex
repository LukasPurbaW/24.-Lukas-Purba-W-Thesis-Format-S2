\begin{center}
\Large\textbf{Kata Pengantar}
\end{center}
\vspace{2ex}
%Tulis kata pengantar di sini
Puji syukur kehadirat Tuhan YME. atas segala limpahan berkat dan rahmat-Nya, penulis dapat menyelesaikan penelitian ini dengan judul \textbf{Deteksi Kerumunan untuk Pencegahan Penularan Virus Corona Berbasis Video Menggunakan Mask-RCNN}.

Penelitan ini disusun dalam rangka pemenuhan bidang riset di Departemen Teknik Elektro, serta digunakan sebagai persyaratan menyelesaikan pendidikan S2. Penelitian ini dapat terselesaikan tidak lepas dari bantuan berbagai pihak. Oleh karena itu, penulis mengucapkan terima kasih kepada:

\begin{enumerate}[nolistsep]
	\item Keluarga, Ibu, Bapak, dan kakak tercinta yang telah memberikan dorongan spiritual dan material dalam penyelesaian penelitian ini.
	\item Bapak Dedet Candra Riawan, ST., M.Eng., Ph.D. selaku Kepala Departemen Teknik Elektro, Fakultas Teknologi Elektro dan Informatika Cerdas (FTEIC), Institut Teknologi Sepuluh Nopoember.
	\item Bapak Reza Fuad Rachmadi, ST., MT., Ph. D selaku dosen pembimbing I yang telah memberikan arahan dan saran selama mengerjakan penelitian tugas akhir ini.
	\item Bapak Dr. Eko Mulyanto Yuniarno, ST., MT. selaku dosen pembimbing II yang telah memberikan arahan dan saran selama mengerjakan penelitian tugas akhir ini.
	\item Bapak Dr. Adhi Dharma Wibawa, ST., MT. dan Bapak Eko Setijadi ST, MT, Ph.D. selaku dosen penguji yang telah memberikan saran dan masukkan yang berguna untuk penulisan penelitian.
	\item Bapak-ibu dosen pengajar Departemen Teknik Elektro atas pengajaran, bimbingan, serta perhatian yang diberikan kepada penulis selama ini.
	\item Seluruh teman - teman dari Telematika 2020, Teknik Elektro yang telah mendukung dan menjadi teman seperjuangan saya.
  
\end{enumerate}

\vspace{26pt}
	\begin{flushright}
		\begin{tabular}[b]{c}
			Surabaya, April 2022
			\\
			\\
			\\
			Penulis
		\end{tabular}
	\end{flushright}
