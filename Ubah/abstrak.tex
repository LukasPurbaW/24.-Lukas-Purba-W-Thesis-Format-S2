\begin{spacing}{1}
\begin{center}
		\large\textbf{\JdTesis}
	\end{center}
	\normalsize
	\begin{adjustwidth}{-0.2cm}{}
		\ifthenelse{\boolean{bMaster}}{
		
		\begin{tabular}{lcp{0.9\linewidth}}
		Nama Mahasiswa &:& \NamaMahasiswa\\
			NRP &:&\NrpMahasiswa\\
			Pembimbing &:& 1. \PbSatu\\
			\ifthenelse{\boolean{PembimbingDua}}{& & 2. \PbDua\\}{}
			
			\ifthenelse{\boolean{PembimbingTiga}}{& & 3. \PbTiga\\}{}
			
			
			
		\end{tabular}
	}{
	\begin{tabular}{lcp{0.7\linewidth}}
		Nama Mahasiswa &:& \NamaMahasiswa\\
		NRP &:&\NrpMahasiswa\\
		Promotor &:&  \PbSatu\\
		\ifthenelse{\boolean{PembimbingTiga}}{
		\ifthenelse{\boolean{PembimbingDua}}{Co. Promotor&: & 1. \PbDua\\}{}
		\ifthenelse{\boolean{PembimbingTiga}}{& & 2. \PbTiga\\}{}
	}
{
	\ifthenelse{\boolean{PembimbingDua}}{\hspace{5ex}Co. Promotor&: & \PbDua\\}{}
	
}
	\end{tabular}

}

	
	\end{adjustwidth}
	\vspace{2ex}
	\begin{center}
		\Large\textbf{ABSTRAK}
	\end{center}
	\vspace{1ex}	
%Tulis Abstrak disini
Pandemi Covid-19 yang berlangsung dari tahun 2020 telah berlangsung selama 2 tahun. Walaupun
banyak masyarakat telah menerima vaksin, pemerintah tetap menegaskan masyarakat untuk melakukan protokol kesehatan
terutama di tempat umum.
Mengikuti saran dari WHO, salah satu protokol kesehatan yang perlu diterapkan adalah menjaga jarak satu sama lain
dengan jarak satu meter.
Namun, dalam penerapan protokol kesehatan, tentunya banyak masyarakat yang lalai untuk menerapkan hal tersebut.
Oleh karena itu, sebuah sistem deteksi yang dapat membantu memudahkan penerapan protokol kesehatan dirasa dibutuhkan.

Penelitian ini memanfaatkan algoritma Mask-RCNN yang digunakan untuk mendeteksi apakah terdapat kerumunan dalam suatu gambar yang diambil.
Objek yang dideteksi pada penelitian ini dibagi menjadi tiga kelas yaitu : Crowd, Group, dan Person. Kelas Crowd merupakan
kelas dimana terletak 4 orang atau lebih dalam radius 1 m, kelas group merupakan kelas dimana terdapat 2-3 orang dalam radius
1 m sedangkan kelas person merupakan kelas dimana hanya terdapat 1 orang dalam radius 1 m.
Mask R-CNN mendeteksi objek dengan cara memberikan beberapa proposal objek yang kemudian diproses dengan layer konvolusi dan \emph{fully connected layer}.
Keluaran dari layer konvolusi adalah \emph{mask} yang menunjukkan segmentasi gambar, sedangkan keluaran dari \emph{fully connected layers} adalah memberikan \emph{bounding box} serta menentukan nama kelas dari objek tersebut.
Hasil dari penelitian ini adalah model yang dapat mendeteksi adanya kerumunan dengan akurasi sebesar 88.79\%.

%Tulis Kata Kunci disini
\vspace{2ex}
\textbf{Kata kunci }: Mask R-CNN, Deteksi Kerumunan, \emph{deep learning}, Covid-19.
	
\end{spacing}